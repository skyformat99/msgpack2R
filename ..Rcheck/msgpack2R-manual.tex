\nonstopmode{}
\documentclass[letterpaper]{book}
\usepackage[times,inconsolata,hyper]{Rd}
\usepackage{makeidx}
\usepackage[utf8,latin1]{inputenc}
% \usepackage{graphicx} % @USE GRAPHICX@
\makeindex{}
\begin{document}
\chapter*{}
\begin{center}
{\textbf{\huge Package `msgpack2R'}}
\par\bigskip{\large \today}
\end{center}
\begin{description}
\raggedright{}
\item[Type]\AsIs{Package}
\item[Title]\AsIs{Convert to and from Msgpack Objects}
\item[Version]\AsIs{0.1}
\item[Date]\AsIs{2017-09-27}
\item[Author]\AsIs{Travers Ching}
\item[Maintainer]\AsIs{Travers Ching }\email{traversc@gmail.com}\AsIs{}
\item[Description]\AsIs{Convert to and from msgpack objects.}
\item[License]\AsIs{GPL-2}
\item[LazyLoad]\AsIs{yes}
\item[Imports]\AsIs{Rcpp (>= 0.11.0), BH}
\item[LinkingTo]\AsIs{Rcpp, BH}
\item[RoxygenNote]\AsIs{6.0.1}
\end{description}
\Rdcontents{\R{} topics documented:}
\inputencoding{utf8}
\HeaderA{msgpack2R}{msgpack2R}{msgpack2R}
\aliasA{msgpack2R-package}{msgpack2R}{msgpack2R.Rdash.package}
%
\begin{Description}\relax
Package for converting to and from msgpack objects
\end{Description}
%
\begin{Author}\relax
Travers <traversc@gmail.com>
\end{Author}
\inputencoding{utf8}
\HeaderA{msgpack\_format}{Format data for msgpack}{msgpack.Rul.format}
%
\begin{Description}\relax
A helper function to format R data for input to msgpack
\end{Description}
%
\begin{Usage}
\begin{verbatim}
msgpack_format(x)
\end{verbatim}
\end{Usage}
%
\begin{Arguments}
\begin{ldescription}
\item[\code{x}] An r object.
\end{ldescription}
\end{Arguments}
%
\begin{Value}
A formatted R object to use as input to msgpack\_pack.
\end{Value}
\inputencoding{utf8}
\HeaderA{msgpack\_map}{Msgpack Map}{msgpack.Rul.map}
%
\begin{Description}\relax
A helper function to create a map object for input to msgpack
\end{Description}
%
\begin{Usage}
\begin{verbatim}
msgpack_map(key, value)
\end{verbatim}
\end{Usage}
%
\begin{Arguments}
\begin{ldescription}
\item[\code{key}] A list or vector of keys (coerced to list).  Duplicate keys are fine (connects to std::multimap in C++).

\item[\code{value}] A list or vector of values (coerced to list).  This should be the same length as key.
\end{ldescription}
\end{Arguments}
%
\begin{Value}
An data.frame also of class "map" that can be used as input to msgpack\_pack.
\end{Value}
\inputencoding{utf8}
\HeaderA{msgpack\_pack}{Msgpack Pack}{msgpack.Rul.pack}
%
\begin{Description}\relax
Serialize any number of objects into a single message.
\end{Description}
%
\begin{Usage}
\begin{verbatim}
msgpack_pack(...)
\end{verbatim}
\end{Usage}
%
\begin{Arguments}
\begin{ldescription}
\item[\code{...}] Any R objects that have corresponding msgpack types.
\end{ldescription}
\end{Arguments}
%
\begin{Value}
A raw vector containing the message.
\end{Value}
%
\begin{SeeAlso}\relax
See test.r for examples in the package directory.
\end{SeeAlso}
\inputencoding{utf8}
\HeaderA{msgpack\_simplify}{Simplify msgpack}{msgpack.Rul.simplify}
%
\begin{Description}\relax
A helper function for simplifying a msgpack return object
\end{Description}
%
\begin{Usage}
\begin{verbatim}
msgpack_simplify(x)
\end{verbatim}
\end{Usage}
%
\begin{Arguments}
\begin{ldescription}
\item[\code{x}] Return object from msgpack\_unpack.
\end{ldescription}
\end{Arguments}
%
\begin{Value}
A simplified return object from msgpack\_unpack.  E.g., arrays of all the same type are concatenated into an atomic vector.
\end{Value}
\inputencoding{utf8}
\HeaderA{msgpack\_unpack}{Msgpack Pack}{msgpack.Rul.unpack}
%
\begin{Description}\relax
De-serialize a msgpack message.
\end{Description}
%
\begin{Usage}
\begin{verbatim}
msgpack_unpack(message)
\end{verbatim}
\end{Usage}
%
\begin{Arguments}
\begin{ldescription}
\item[\code{message}] A raw vector containing the message.
\end{ldescription}
\end{Arguments}
%
\begin{Value}
The message pack object(s) converted into R types.  If more than one object exists in the message, a list of class "msgpack\_set" containing the objects.
\end{Value}
%
\begin{SeeAlso}\relax
See test.r for examples in the package directory.
\end{SeeAlso}
\printindex{}
\end{document}
